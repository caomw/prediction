\section{Source Separation Algorithm}
\label{algosec}
We show in this section how the inverse problem of source separation 
can be solved via a sparse coding in the scattering domain, followed by phase recovery.

We consider the supervised 
monoaural source separation problem, in which one observes mixtures 
\begin{equation}
\label{ssep}
x(t) = x_1(t) + x_2(t)~,
\end{equation}
where $x_i$ come from sources from which we assume training data $X_i=\{x_{ij}\}_{j \leq K}$, 
and one is asked to produce estimates $\widehat{x_i}$. 
For example, we might consider a male-vs-female separation task, by 
collecting male and female training examples. 

The supervision provides  prior information on the nature of each of the 
components. However, high-dimensional speech signals have large variability, 
most of which is uninformative for the purposes of estimating $x_i$ in (\ref{ssep}).
The training data can be exploited more efficiently in the scattering domain, since 
intra-class variability given by small pitch and timber variations is linearized up to 
larger temporal scales without loosing as much discriminative information as 
the spectrogram \cite{deepscatt,pami}.

Let $\Phi(X_i)$ be the scattering representation of the training examples of each 
source. We consider a non-linear approximation of each source using non-Negative
matrix factorization:
\begin{equation}
\min_{D_i\geq 0, Z_i\geq 0} \frac{1}{2} \| \Phi(X_i) - D_i \, Z_i \|_F^2 + \lambda \| Z_i \|_1~.
\end{equation}
This model exploits the linearization properties of scattering coefficients since it 
searches low-dimensional locally linear approximations. 

At test time, given and input $x$, $x_1$ and $x_2$ are estimated by minimizing
\begin{equation}
\label{model}
\min_{x'_i, z_i} \sum_{i=1,2} \frac{1}{2} \| \Phi(x'_i) - D_i z_i \|_2^2 + \lambda \| z_i \|_1 \quad\,s.t. ~x=x'_1 + x'_2~.
\end{equation}
Problem (\ref{model}) is minimized with an alternating gradient descent between $x'_i$ and $z_i$. 
Fixing $z_i$ and minimizing with respect to $x'_i$ requires locally inverting the scattering 
operator $\Phi$, which amounts to solve an overcomplete phase recovery problem and 
can be solved with gradient descent, as shown in \cite{bruna2013audio}. 
Fixed $x'_i$, solving for $z_i$ is a standard $\ell_1$ sparse coding problem, which can be solved 
efficiently with proximal splitting algorithms. In this work, we use the LARS algorithm. 
 
%Explain the greedy algorithm.
When the analysis operator $\Phi$ is able to produce sparse representations of the sources, 
then 
\begin{eqnarray*}
\| \Phi(x'_1) - D_1 z_1 \|_2^2 + \| \Phi(x'_2) - D_2 z_2 \|^2 &\approx& \\
\| \Phi(x'_1) + \Phi(x'_2) - \sum_{i=1,2} D_i z_i \|_2^2 &\approx& \\
 \| \Phi(x) - D_1 z_1 - D_2 z_2 \|_2^2 ~,
\end{eqnarray*}
which can be used in practice to produce a greedy initialization for (\ref{model}) as follows. 
We first obtain $\widehat{\Phi(x_i)}= D_i z^*_i$, where the $z^*_i$ are defined as 
$$z^*_i = \arg\min_{z_i} \frac{1}{2}\| \Phi(x) - \sum_{i=1,2} D_i z_i \|_2^2 +\lambda \| z_i \|_1~.$$
Since the scattering $\Phi(x)$ is defined as \\$\Phi(x)= \{ A |W_1 x | \,,\, A | W_2 | W_1 x | | \}$, 
where $A$ is the lowpass filter and $W_1$ and $W_2$ are respectively the first and 
second layer wavelet decompositions, we can produce an estimate $\widehat{x}_i$ from 
$\widehat{\Phi(x_i)}$ by using the complex phases of $W_1 x$ and $W_2 |W_1 x|$.



%
%
%\subsection{Analysis operator}
%
%Common choices are the (non-negative) spectrogram or the the constant $Q-$transform \cite{}. 
%
%Given the observed noisy signal $x$, we denote it's spectrogram as $\bb{\Phi}(x) = |\bb{X}| \in \RR^{m \times n}$ comprising $m$ frequency bins and $n$ temporal frames. $\bb{X} \in \CC^{m \times n}$ contains in it's $i-$th column the Discrete Fourier Transform (DFT) of the $i-th$ frame of $x$, $\bb{x}_i\in \RR^{m}$ ,and it is given by, $\bb{X}_i = \bb{W}^\Tr \bb{x}_i$.
%
%
