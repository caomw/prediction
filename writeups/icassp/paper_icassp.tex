% Template for ICASSP-2015 paper; to be used with:
%          spconf.sty  - ICASSP/ICIP LaTeX style file, and
%          IEEEbib.bst - IEEE bibliography style file.
% --------------------------------------------------------------------------
\documentclass{article}
\usepackage{spconf,amsmath,graphicx}

\usepackage{anyfontsize}

\usepackage{fixltx2e}
\usepackage{overpic}
\usepackage{multirow}
\usepackage{graphicx} % more modern
%\usepackage{epsfig} % less modern
\usepackage{subfig}
\usepackage{sidecap}
\usepackage{url}
\usepackage[applemac]{inputenc}
\usepackage{pst-sigsys}
\usepackage{tabularx}
\usepackage{xfrac}
\usepackage{float}
\restylefloat{table}

% For algorithms
\usepackage{algorithm}
\usepackage{algorithmic}

\usepackage{amsmath}
\usepackage{amssymb}
\usepackage{color}
\usepackage{subfig}
%\usepackage[font=small,labelfont=bf]{caption}

% Example definitions.
% --------------------
\def\x{{\mathbf x}}
\def\Vn{{\cal V}}

\def\Has{\bb{H}_{\mathrm{s}}^\ast(\bb{V},\bb{W})}


\def\Tr{\mathrm{T}}

\newcommand{\N}{\mathcal{N}}
\newcommand{\ele}{l}
\newcommand{\X}{\mathbf{X}}
\newcommand{\R}{\mathbb{R}}
\newcommand{\RR}{\mathbb{R}}
\newcommand{\CC}{\mathbb{C}}
%\newcommand{\red}[1]{{#1}}
%\newcommand{\red}[1]{\textcolor{red}{#1}}
\newcommand{\tr}{\mbox{tr}}

\newcommand {\MEL} {{\mathrm{M}}}
\newcommand {\Z} {{\mathbb{Z}}}
\newcommand {\E} {{\mathbb{E}}}
\newcommand {\om} {{\omega}}
\newcommand {\la} {{\lambda}}
\newcommand {\La} {{\Lambda}}
\newcommand{\hS}{\widehat S}
\newcommand {\ga} {{\gamma}}
\newcommand {\MFSC} {{\rm MFSC}}
\newcommand {\oS} {{\overline S}}
\newcommand {\lau} {{\lambda_1}}
\newcommand {\lad} {{\lambda_2}}
\newcommand {\lam} {{\lambda_m}}
\newcommand {\Omu} {{\Omega_1}}
\newcommand {\Omd} {{\Omega_2}}
\newcommand {\Omm} {{\Omega_m}}
\newcommand {\omu} {{\omega_1}}
\newcommand {\omd} {{\omega_2}}
\newcommand {\omm} {{\omega_m}}
\newcommand {\quef} {{q}}
\newcommand {\quefu} {{q_1}}
\newcommand {\quefd} {{q_2}}
\newcommand {\lat} {{\lambda_3}}
\newcommand {\cP} {{\cal P}}
\newcommand {\cPP} {{\cal P}}
\newcommand {\bV} {{\bf V}}
\newcommand {\C} {{\mathbb C}}
\newcommand {\mm} {{l}}
\newcommand {\tS} {{\widetilde{S}}}
\newcommand {\Ufreq} {{U^{\mathrm{fr}}}}
\newcommand {\Sfreq} {{S^{\mathrm{fr}}}}
\newcommand {\freq} {{{\mathrm{fr}}}}


\usepackage{bm}
\newcommand{\bb}[1]{\bm{\mathrm{#1}}}


\newcommand{\mypara}[1]{\smallskip\noindent\textbf{#1.\,\,}}

% Title.
% ------
\title{Source Separation with Scattering Non-Negative Matrix Factorization}
%
% Single address.
% ---------------
\name{Joan Bruna, Pablo Sprechmann and Yann LeCun}
\address{New York University\\
	Courant Instittute of Mathematical Sciences\\
	\{bruna, pablo\}@cims.nyu.edu, yann@cs.nyu.edu}
%
% For example:
% ------------
%\address{School\\
%	Department\\
%	Address}
%
% Two addresses (uncomment and modify for two-address case).
% ----------------------------------------------------------
%\twoauthors
%  {Joan Bruna, Pablo Sprechmann}
% {New York University\\
%%	Courant Instittute of Mathematical Sciences\\
%	\{bruna, pablo\}@cims.nyu.edu}
%  {Yann Lecun}
%	{New York University and Facebook Inc.\\
%%	Courant Instittute of Mathematical Sciences\\
%	yann@cs.nyu.edu}
%
\begin{document}
\ninept
%
\maketitle
%
\begin{abstract}
This paper presents a single-channel source separation method
that extends the ideas of Nonnegative Matrix Factorization (NMF).
We interpret the approach of audio demixing via NMF as a cascade of a pooled analysis operator, given
for example by the magnitude spectrogram, and a synthesis operators given by the matrix decomposition.
Instead of imposing the temporal consistency of the decomposition through
sophisticated structured penalties in the synthesis stage,
we propose to change the analysis operator for a deep scattering representation,
where signals are represented at several time resolutions. 
This new signal representation is invariant to smooth changes in the signal, consistent with its temporal dynamics.
We evaluate the proposed approach in a speech separation task obtaining promising results.
\end{abstract}
%
\begin{keywords}
source separation, scattering, non-negative matrix factorization.
\end{keywords}
%

\section{Introduction}

The problem of identifying and extracting different speech signals 
recorded in a single, noisy environment has been widely
studied in the audio processing community \cite{loizou2007speech,hansler2008speech}. 
It becomes particularly challenging when only one microphone is used, or in the presence of 
non-stationary background noise, which is a very common situation in many applications encountered, e.g., in telephony.
We approach this problem as a monaural source separation method
by modeling the speech at an appropriate temporal resolution.
%This is a natural approach when the characteristics of both the signal of interest and the noise vary throughout time \cite{WilsonRSD08,JoderWEVS12,MysoreS11,DuanMS12}.
%Consequently, many works have used source separation techniques for addressing this problem,

The decomposition of time-frequency representations, such as the power or magnitude spectrogram
in terms of elementary atoms of a dictionary, has become a popular tool in audio processing. 
Non-negative matrix factorization (NMF) \cite{NMF, smaragdis2006probabilistic},
have been widely adopted in various audio processing tasks, including in particular source separation, see \cite{smaragdis2014static} for a recent review. 
There are many works that follow this line in speech separation \cite{schmidt06speechseparation,shashanka_icassp07} and enhancement \cite{JoderWEVS12,DuanMS12,schmidt07mlsp,mohammadiha2013supervised}. %and robust automatic speech recognition \cite{GemmekeVH11,WeningerWGSGHVR12},
%among many others.
%bandwidth extension \cite{BansalRS05,HanMP12} and speaker recognition \cite{wu2010robust,joder2012exploring}.

%NMF and PLCA produce high quality separation results when the dictionaries
%for different sources are sufficiently distinct.
%There is naturally a compromise between the approximation
%of the training data and tightness of the model: the more general is the dictionary the higher is the chance it will include elements that
%match spectral patterns in the competing sources.
%In order to mitigate this problem, recent approaches have proposed
%alternative models constraining the solution in meaningful ways,
%as for example, by imposing sparsity of the activations \cite{shashanka_icassp07,hoyer2004non}.
%

In plain NMF, signals that can be well approximated with the learned dictionary are
likely to resemble the training data on a frame by frame manner. They might, however, 
not be temporally consistent at larger temporal scales.
 Standard NMF approaches treat different time-frames independently, ignoring the 
temporal dynamics of the signals. In other words, there is additional structure in speech at a time-scale larger than
the frame-length that cannot be learned (or exploited) with NMF. In order to overcome this limitation,  
%and achieve consistency at a larger temporal scale,
many works have proposed regularized extensions of NMF to promote learned (or designed) structure in the codes. Examples of these approaches are, temporal smoothness of the activation coefficients \cite{fevotte2011majorization}, including  co-occurrence statistics of the basis functions \cite{WilsonRSD08}, and learned temporal dynamics \cite{MysoreS11,HanMP12,icassp13a,mohammadiha2013nonnegative}.
%Recent studies have proposed regularized variants of NMF or PLCA trying to overcome this limitation,

NMF-based source separation methods can be thought as the concatenation of two operators.
First, the signal is represented in a feature space given by a non-linear analysis operator, 
typically defined as the magnitude of a time-frequency representation such
as the Short-Time Fourier Transform (STFT). 
Then a synthesis operator, given by the dictionary learning stage, is applied to produce an unmixing in the feature space.
Finally, the separation is obtained by inverting these representations. 
Performing the separation in the non-linear representation is key to the success of the algorithm. The magnitude STFT is in general sparse (simplifying the separation process) and invariant to variations in the phase, thus relieving the NMF from
learning this irrelevant variability. 
This comes at the expense of inverting the unmixed estimates in the feature space, normally known as the
phase recovery problem \cite{yonina}. In the case of standard NMF, this is easily done via Wiener filtering technique discussed in Section \ref{sec:nmf}.

In this work, rather than seeking for a coding scheme with temporal regularity, we seek to encode a representation of the audio signal at a larger scale in which natrual variability in speech is highly compressible. 
Increasing the temporal scale of STFT or MEL representations either produces unstabilities due to 
variations in pitch and timbre, or looses important discriminative information \cite{deepscatt}. 
In order to overcome the limitations of these shallow representations, 
scattering transforms \cite{pami, deepscatt} cascade several 
stages of complex wavelet decompositions and complex modulus, achieving 
discriminative representations with the ability to capture temporal structures at larger scales, 
e.g. smooth changes in pitch and envelope. 
Scattering transforms achieve state-of-the-art results on auditory texture discrimination, and music genre recognition \cite{deepscatt, phdjoan}.
A dictionary learnt to represent the signal in this deep representation would implicitly be learning the short term temporal dynamics of the signal.

%Scattering transforms have recently been introduced \cite{XX} to represent audio signals and images, A scattering transform iterates on complex wavelet transforms and modulus operators which compute their envelop. It has close relations with psychophysical and physiological models \cite{XX}. %Their success in audio classification comes from the fact that they can reduce the intra class variability (like temporal redundancy).

Our claim is that an important part of the consistency that is imposed via structured NMF, can be eliminated with a better signal representation.
In this new setting one can learn the temporal dynamics with a very simple NMF encoding. However, the problem that becomes
more difficult is that of inverting the representation. Recent studies in textured sound synthesis from scattering
moments have solved this problem successfully using gradient descent algorithms \cite{bruna2013audio}.

Synthesis models with coherent dictionaries are known to be highly unstable representations \cite{jenatton2012local}. 
Thus, training them to satisfy slowness and temporal consistency can be challenging. For example, in \cite{icassp13a}
the authors explain that the learning the temporal dynamics in NMF via a Kalman type of model, can become very difficult
when the coding is sparse due to the instability and jitter in the codes.
In contrast, analysis operators are stable by construction.

Section \ref{nmfsec} reviews non-negative matrix factorization, while section \ref{scattsec} describes scattering 
representations for speech. Our source separation algorithm is described in Section \ref{algosec} and 
numerical experiments on TIMIT and Grid datasets are reported in Section \ref{resultssec}.

%In all these methods the audio signal is represented as a non-negative and the demixng is obtained  model is expressed as the minimization of a cost with a data fitting term and some structure-promoting penalties.




\section{NMF speech source separation}
\label{nmfsec}

%
We consider the setting in which we are given a temporal signal $x(t)$ that is the sum of  
two speech signals $x_i(t)$, $i=1,2$:
\begin{equation}
\label{ssep}
x(t) = x_1(t) + x_2(t)~,
\end{equation}
and we aim at finding estimates $\widehat{x_i}(t)$.
NMF-based source separation 
techniques typically operate on a non-negative time-frequency representation of $x(t)$, 
such as the spectrogram or the power spectrum,
that we denote as $\Phi(x) \in \RR^{m \times n}$, comprising $m$ frequency bins and $n$ temporal frames. 
NMF attempts to find the non-negative activations $Z_i \in \RR^{q \times n}$, $i=1,2$ 
best representing the different speech components in two dictionaries ${D}_{i} \in \RR^{m \times q}$.
%
This task is achieved through the solution of %the minimization problem
\begin{eqnarray}
\label{eq:optim_general}
\min_{ Z_i \ge 0 } \mu( \Phi(x) |  \, \sum_{i=1,2} {D}_i Z_i  ) + 
\lambda\, \sum_{i=1,2} \mathcal{R}(Z_i)~.
\end{eqnarray}
The first term in the optimization objective measures the dissimilarity between the input data and the estimated channels. 
%Typically, this data fitting term is assumed to be separable,  
%$$
%D( \bb{A} | \bb{B} ) = \sum_{i,j} D(a_{ij} | b_{ij}).
%$$ 
Frequent choices of $\mu$ are the squared Euclidean distance,
the Kullback-Leibler divergence, and the Itakura-Saito divergence, for which there exist standard optimization algorithms \cite{fevotte2011algorithms}.
In this work we concentrate on a reweighted Euclidean distance, but any other option could be used instead.
%
%Significant attention has been devoted in the literature to the case in which the scalar divergence $D$
%belongs to the family of the so-called $\beta$-divergences, 
%\begin{equation*}
%D_{\beta} (a|b) =\!\!\left \{
%\begin{array}{ll}
%\!\frac{a}{b} - \log{\frac{a}{b}}-1 & \! : \,\beta = 0,\\
%\! a \log{a/b} + (a-b) &\! : \, \beta = 1,\\
%\!\frac{1}{\beta(\beta-1)} (a^{\beta} +(\beta-1)b^{\beta} -\beta a b^{\beta-1}) &\! : \textrm{otherwise.}\\
%\end{array}
%\right.
%\end{equation*}
%This family includes 
%where the three most widely used cost functions in NMF: the squared Euclidean distance ($\beta=2$),
%the Kullback-Leibler divergence ($\beta=1$), and the Itakura-Saito divergence ($\beta=0$).
%For $\beta \ge 1$, the divergence is convex. The case of $\beta=0$ is attractive despite the lack of convexity, due to the scale-invariance of the Itakura-Saito divergence, which makes the NMF procedure insensitive to volume changes
%
The second term in the minimization objective is included to promote some desired structure of the activations. 
Once the optimal activations are solved for, the spectral envelopes of the speech and the noise are 
estimated as $ {D}_{i} Z_{i}$. Since these estimated speech spectrum envelope contain no phase information, 
they are used to build soft masks to filter the mixture signal \cite{schmidt07mlsp}.
%This is done using a designed regularization function $\mathcal{R}$ 
%and its relative importance is controlled by the parameters $\lambda$. 
%Since these estimated speech spectrum envelope contains no phase information, speech signal is estimated
%from the mixture by Wiener filtering. 


%In supervised NMF, the speech and noise dictionaries are trained independently from available training data. 
%
%The underlying assumption of this approach is that the speech and the noise signals forming the mixture are sufficiently distinct to be unambiguously decomposed 
%into $\bb{V} \approx  \bb{W}_{\mathrm{s}} \bb{H}_{\mathrm{s}} + \bb{W}_{\mathrm{n}}\bb{W}_{\mathrm{n}}$. 
%However, this assumption is often violated, e.g., in the presence of multitalker babble noise, when the learned speech and noise dictionaries might be very similar (or coherent).
%
%In other words, the independently trained dictionaries do not ensure that the solutions $\bb{W}_{\mathrm{s}} \bb{H}_{\mathrm{s}}$ and $\bb{W}_{\mathrm{n}} \bb{H}_{\mathrm{n}}$ obtained from \eqref{eq:optim} will resemble the original components of the mixture. 

%\section{Scattering transform}
\label{scattsec}

Discriminative features having longer temporal context can be constructed with the 
scattering transform \cite{pami, deepscatt}. 
This section reviews its definition and main properties when applied to speech signals.

\subsection{Wavelet Filter Bank}

A wavelet $\psi (t)$ is a band-pass filter with good frequency and spatial localization.
 We consider a complex wavelet with a quadrature phase, 
whose Fourier transform satisfies
$\widehat \psi(\om) \approx 0$ for $\om < 0$.
We assume that the center frequency of $\widehat \psi$ is $1$ and 
that its bandwidth is of the order of $Q^{-1}$. 
Wavelet filters centered
at the frequencies $\lambda = 2^{j/Q}$ are computed by dilating $\psi$:
\begin{equation}
\label{psidilation}
\psi_\la (t) = \la\, \psi(\la\, t)~~\mbox{and hence}~~
\widehat \psi_\la (\om) = \widehat \psi(\la^{-1} \omega)~.
\end{equation}
We denote by $\Lambda$ the index set of $\la = 2^{j/Q}$ over
the signal frequency support, with $j \leq J$, 
and we impose that these filters fully cover the positive frequencies:
\begin{equation}
\label{consdf}
\forall \om \geq 0~,~ 1- \epsilon 
\leq |\widehat{\phi}(\omega)|^2 + \frac 1 2 \sum_{\la \in \Lambda} |\widehat{\psi}_{\lambda}(\omega)|^2 \leq 1~,
\end{equation}
for some $\epsilon <1$, where $\phi(t)$ is the lowpass filter carrying the 
low frequency information at scales larger than $2^J$.
The resulting filter bank has a constant number $Q$ of bands per 
octave. The wavelet transform of a signal $x(t)$ is
\[
W x = \{x \ast \phi(t)~,~x \ast \psi_\la(t)  \}_{\la \in \Lambda}~.
\]
Thanks to (\ref{consdf}), one can verify that 
\begin{equation}
\label{lp_basic}
\| x \|^2 (1- \epsilon) \leq \| x \ast \phi \|^2 + \sum_{\la \in \Lambda} \| x \ast \psi_\la \|^2  \leq \| x \|^2~.
\end{equation}

\subsection{Joint Time-Frequency Scattering}

Scattering coefficients provide a nonlinear representation 
computed by 
iterating over wavelet transforms and a modulus. 
First order scattering coefficients
are local averages of wavelet coefficient amplitudes:
\[
\forall \la \in \La~~,~~S x(t,\la) = |\, x \ast \psi_\la| \ast \phi(t)~. 
\]
The Q-factor $Q_1$ adjusts the frequency resolution of 
these wavelets, and for speech it is typically around $Q_1=32$.
 Due to the temporal average, first order scattering coefficients 
provide no information on the time-variation of the scalogram
$|x \ast \psi_{\la_1} (t)|$ at temporal scales smaller than $2^J$. 
It averages all modulations and 
transient events, and thus lose perceptually important information.
 
Second order scattering coefficients recover information on
audio modulations and pitch temporal variations by computing
the wavelet coefficients of the envelopes $|x \ast \psi_{\la_1}|$, and their
local averages:
\[
\forall \la_2~~,~~
S x(t,\la_1,\la_2) = ||\, x \ast \psi_{\la_1}| \ast \psi_{\la_2}| \ast \phi(t)~ .
\]
These multiscale variations of each envelope $|x \ast \psi_{\la_1}|$ 
specify the amplitude modulations of $x(t)$ \cite{deepscatt} and 
thus have the capacity to detect rythmic structures appearing at different 
frequency bands. 
The Q-factor $Q_2$ of the second family of wavelets $\psi_{\la_2}$ 
controls the time-frequency resolution of the transform. Smaller $Q_2$ 
results in wavelets with good temporal resolution, and thus
allows us to accurately measure the sharp transitions of
amplitude modulations. 
Scattering coefficients have a negligible amplitude for
$\la_2 > \la_1$ because $|x \ast \psi_{\la_1}|$ is a regular envelop
whose frequency support is below $\la_2$ \cite{pami}. 
Scattering coefficients are thus computed only for $\la_2 < \la_1$. 

%
%
%Applying more wavelet transform envelopes defines
%scattering coefficients at any order $m \geq 1$:
%\begin{equation}
%\label{expansdf}
%S I(\la_1,...,\la_m; t) =  |~|I \ast \psi_{\la_1}| \ast ... | \ast \psi_{\la_m}| \ast \phi(t)~.
%\end{equation}
%By iterating on the inequality (\ref{lp_basic}), one can 
%verify \cite{stephane}
%that the Euclidean norm of scattering coefficients
%\begin{equation}
%\label{enfsondf}
%\|S I\|^2 = \sum_{m=1}^{\infty} \sum_{(\la_1,...,\la_m) \in \La_m}
%\|S I (\la_1,...,\la_m; \cdot )\|^2 
%\end{equation}
%satisfies
%\[
%\|S I\|^2 \leq  \| I \|^2~.
%\]
%
%For most audio signals, the energy of the scattering
%vector $\|S I\|^2$ is concentrated over first and second layers. 
%In practice, we thus only compute $S I(\la_1)$ and
%$S I(\la_1,\la_2)$ for $1 \leq \la_1 = 2^{j_1/Q_1} \leq N$
%and $1 \leq \la_2 = 2^{j_2/Q_2} < \la_1$. 
%
Scattering transforms have been extended along the frequency
variables to capture the joint time-frequency 
variability of spectral envelopes and therefore provide 
 representations locally stable to pitch variations \cite{deepscatt}. 
We denote $\gamma = \log_2 \lambda_1$, and consider
the scalogram as a two-dimensional function of $\gamma$ and $t$: 
\[
F (\gamma, t) = |x \ast \psi_{2^{\gamma}} (t)|~.
\]

In this work, we consider a second layer scattering with 
a separable wavelet transform $F \ast \Psi_{\gamma_2, \la_2} (\gamma, t)~,$
with
$$\Psi_{{\gamma_2}, {\la_2}}(\gamma,t) = \widetilde{\psi}_{\gamma_2}(\gamma) \psi_{\la_2}(t)~.$$
The temporal wavelets $\psi(t)$ are dyadic complex Morlet wavelets. 
In this implementation, 
we choose $\widetilde{\psi}$ to be dyadic real Haar wavelets to preserve good frequency localization.

The resulting second order scattering coefficients are thus
\[
\tilde{S} x (t, \la_1, \gamma_2,  \la_2) = 
 |F \ast \bar \Psi_{\gamma_2, \la_2} | \ast \overline{\phi}( \log_2 \la_1, t)~,
\]
where $\overline{\phi}(\gamma, t)$ is a two-dimensional blurring kernel with temporal 
scale $2^J$ and log-frequency scale $2^{J_h}$.
 The final representation regrouping first and second order scattering
coefficients and sampling at intervals $k\,2^{J-\delta}$
is $\Phi(x) = \{ Sx(k 2^{J-\delta} k,\la_1)\,,\, \tilde{S}x(k 2^{J-\delta} ,\la_1,\gamma_2, \la_2)\} $, 
where the oversampling factor $\delta$ is typically set to $\delta=1$.


%
%
%\subsection{Analysis operator}
%
%Common choices are the (non-negative) spectrogram or the the constant $Q-$transform \cite{}. 
%
%Given the observed noisy signal $x$, we denote it's spectrogram as $\bb{\Phi}(x) = |\bb{X}| \in \RR^{m \times n}$ comprising $m$ frequency bins and $n$ temporal frames. $\bb{X} \in \CC^{m \times n}$ contains in it's $i-$th column the Discrete Fourier Transform (DFT) of the $i-th$ frame of $x$, $\bb{x}_i\in \RR^{m}$ ,and it is given by, $\bb{X}_i = \bb{W}^\Tr \bb{x}_i$.
%
%

\section{Scattering transform}
\label{scattsec}

Discriminative features having longer temporal context can be constructed with the 
scattering transform \cite{deepscatt,pami}. 
While these features have shown excellent performance in various classification tasks, 
in the context of source separation we require a representation that not only captures
long-range temporal structures, but also preserves as much discriminability as possible.
For this reason, we construct a multi-level representation consisting of a pyramid of scattering coefficients with different temporal resolutions at each level. 
This section reviews its definition and main properties when applied to speech signals.

\subsection{Wavelet Filter Bank}

A wavelet $\psi (t)$ is a band-pass filter with good frequency and spatial localization.
 We consider a complex wavelet with a quadrature phase, 
whose Fourier transform satisfies
$\mathcal{F} \psi(\om) \approx 0$ for $\om < 0$.
We assume that the center frequency of $\mathcal{F} \psi$ is $1$ and 
that its bandwidth is of the order of $Q^{-1}$. 
Wavelet filters centered
at the frequencies $\lambda = 2^{j/Q}$ are computed by dilating $\psi$:
$\psi_\la (t) = \la\, \psi(\la\, t)$, and hence $\mathcal{F} \psi_\la (\om) = \widehat \psi(\la^{-1} \omega)$.
We denote by $\Lambda$ the index set of $\la = 2^{j/Q}$ over
the signal frequency support, with $j \leq J$, 
and we impose that these filters fully cover the positive frequencies:
\begin{equation}
\label{consdf}
\forall \om \geq 0~,~ 1- \epsilon 
\leq |\mathcal{F}{\phi}(\omega)|^2 + \frac 1 2 \sum_{\la \in \Lambda} |\mathcal{F}{\psi}_{\lambda}(\omega)|^2 \leq 1~,
\end{equation}
for some $\epsilon <1$, where $\phi(t)$ is the lowpass filter carrying the 
low frequency information at scales larger than $2^J$.
The resulting filter bank has a constant number $Q$ of bands per 
octave. The wavelet transform of a signal $x(t)$ is
\[
W x = \{x \ast \phi(t)~,~x \ast \psi_\la(t)  \}_{\la \in \Lambda}~.
\]
Thanks to (\ref{consdf}), one can verify that 
\begin{equation}
\label{lp_basic}
\| x \|^2 (1- \epsilon) \leq \| x \ast \phi \|^2 + \sum_{\la \in \Lambda} \| x \ast \psi_\la \|^2  \leq \| x \|^2~.
\end{equation}

\subsection{Joint Time-Frequency Pyramid Scattering}

Scattering coefficients provide a nonlinear representation 
computed by iterating over wavelet transforms and complex modulus nonlinearities. 
%We sample the scalograms at critical rate given by . Note that this contrasts with standard scattering transform, were a fixed bandwidth smoothing kernel is used at every layer \cite{deepscatt,pami}.  
%
%First order scattering coefficients
%are local averages of wavelet amplitudes:
%\[
%\forall \la \in \La~~,~~S x(t,\la) = |\, x \ast \psi_\la| \ast \phi(t)~. 
%\]
%The Q-factor $Q_1$ adjusts the frequency resolution of 
%these wavelets, and for speech it is typically around $Q_1=32$.
% Due to the temporal average, first order scattering coefficients 
%provide no information on the time variation of the scalogram
%$|x \ast \psi_{\la_1} (t)|$ at temporal scales smaller than $2^J$. 
%It averages all modulations and 
%transient events, and thus loses perceptually important information.
We start by removing the complex phase of wavelet coefficients in $W^1x$ with a 
complex modulus nonlinearity. We arrange these first layer coefficients as nodes in the first level of the tree,
\[
|W^1| x = \{ x^1_i \}_{i=1\dots 1+|\Lambda|}= \{x \ast \phi_1(\Delta_1 n)~,~|x \ast \psi_{1,\lambda_1}(\Delta_1 n)|  \}_{\lambda \in \Lambda}~.
\]
where $\Delta_1$ is the critical sampling rate of the highest frequency wavelet sub-band 
(the reciprocal of the largest bandwidth present
in the filter bank) and $\psi_1$ has bandwidth $Q_1^{-1}$.
These first layer coefficients give localized information both in
time and frequency, with a trade-off dictated by the Q factor, $Q_1$, that adjusts the frequency resolution of 
these wavelets.  For speech a typical choice is around $Q_1=32$.

In order to increase the robustness of the representation, we transform each of the down sampled signals from this first layer
with a new wavelet filter bank and take the complex modulus of the oscillatory component. 
In order to sample each channel using the same temporal resolution, this time we apply the lowpass anti-aliasing filter to the demodulated channels:
%For simplicity, we assume a dyadic transformation, 
%which reduces the filter bank to a pair of conjugate mirror filters $\{ \phi_2, \psi_2\} $ \cite{wavelettour}, 
%carrying respectively the low-frequencies and high-frequencies of the discrete signal from above the tree:
\begin{equation}
\label{scatnonjoint}
|W^2| x = \{ x^1_i \ast \phi_{2}  (\Delta_2 n)~,~| x^1_i \ast \psi_{2,\lambda_2} | \ast \phi_{2}(\Delta_2 n)| \}_{i=1\dots |W^1|}~,
\end{equation}
where $\Delta_2$ is this time the critical sampling rate of the averaging filter $\phi_2$. 
The multiscale variations of each envelope specify the amplitude modulations of $x(t)$ \cite{deepscatt} and  thus have the capacity to detect rhythmic structures appearing at different frequency bands. The Q-factor $Q_2$ of the second family of wavelets $\psi_{2,\la_2}$ 
controls the time-frequency resolution of the transform. 
Since the envelopes $| x \ast \psi_{\la_1}|$ have bandwidth $\sim 2^{-j} Q_1^{-1}$, 
one typically chooses dyadic $Q_2=1$ second order wavelets.
Scattering coefficients have a negligible amplitude for
$\la_2 > \la_1$ because $|x \ast \psi_{\la_1}|$ is a regular envelop
whose frequency support is below $\la_2$ \cite{pami}. 
Scattering coefficients are thus computed only for $\la_2 < \la_1$. 

%Second order scattering coefficients recover information on
%audio modulations and pitch temporal variations by computing
%the wavelet coefficients of the envelopes $|x \ast \psi_{\la_1}|$, and their
%local averages:
%\[
%\forall \la_2~~,~~
%S x(t,\la_1,\la_2) = ||\, x \ast \psi_{\la_1}| \ast \psi_{\la_2}| \ast \phi(t)~ .
%\]
%These multiscale variations of each envelope $|x \ast \psi_{\la_1}|$ 
%specify the amplitude modulations of $x(t)$ \cite{deepscatt} and 
%thus have the capacity to detect rythmic structures appearing at different 
%frequency bands. 
%The Q-factor $Q_2$ of the second family of wavelets $\psi_{\la_2}$ 
%controls the time-frequency resolution of the transform. 
%Since the envelopes $| x \ast \psi_\la|$ have bandwidth $\sim 2^{-j} Q_1^{-1}$, 
%one typically chooses dyadic $Q_2=1$ second order wavelets.
%Scattering coefficients have a negligible amplitude for
%$\la_2 > \la_1$ because $|x \ast \psi_{\la_1}|$ is a regular envelop
%whose frequency support is below $\la_2$ \cite{pami}. 
%Scattering coefficients are thus computed only for $\la_2 < \la_1$. 

%
%
%Applying more wavelet transform envelopes defines
%scattering coefficients at any order $m \geq 1$:
%\begin{equation}
%\label{expansdf}
%S I(\la_1,...,\la_m; t) =  |~|I \ast \psi_{\la_1}| \ast ... | \ast \psi_{\la_m}| \ast \phi(t)~.
%\end{equation}
%By iterating on the inequality (\ref{lp_basic}), one can 
%verify \cite{stephane}
%that the Euclidean norm of scattering coefficients
%\begin{equation}
%\label{enfsondf}
%\|S I\|^2 = \sum_{m=1}^{\infty} \sum_{(\la_1,...,\la_m) \in \La_m}
%\|S I (\la_1,...,\la_m; \cdot )\|^2 
%\end{equation}
%satisfies
%\[
%\|S I\|^2 \leq  \| I \|^2~.
%\]
%
%For most audio signals, the energy of the scattering
%vector $\|S I\|^2$ is concentrated over first and second layers. 
%In practice, we thus only compute $S I(\la_1)$ and
%$S I(\la_1,\la_2)$ for $1 \leq \la_1 = 2^{j_1/Q_1} \leq N$
%and $1 \leq \la_2 = 2^{j_2/Q_2} < \la_1$. 
%
Scattering transforms have been extended along the frequency
variables to capture the joint time-frequency 
variability of spectral envelopes and therefore provide 
 representations locally stable to pitch variations \cite{deepscatt}. 
We denote $\gamma = \log_2 \lambda_1$, and consider
the scalogram as a two-dimensional function of $\gamma$ and $t$: 
\[
F (\gamma, t) = |x \ast \psi_{2^{\gamma}} (t)|~.
\]

In this work, we consider a second layer scattering with 
a separable wavelet transform $F \ast \overline{\psi}_{\gamma_2, \la_2} (\gamma, t)~,$
with
$$\overline{\psi}_{{\gamma_2}, {\la_2}}(\gamma,t) = \widetilde{\psi}_{\gamma_2}(\gamma) \psi_{2,\la_2}(t)~,~\overline{\psi}_{{0}, {\la_2}}(\gamma,t) = \widetilde{\phi}(\gamma) \psi_{2,\la_2}(t)~$$
$$\overline{\psi}_{{\gamma_2}, {0}}(\gamma,t) = \widetilde{\psi}_{\gamma_2}(\gamma) \phi_{2}(t)~,~\overline{\phi}(\gamma,t) = \widetilde{\phi}(\gamma) \phi_{2}(t)~$$
By replacing $\psi$ and $\phi$ in (\ref{scatnonjoint}) by $\overline{\psi}$ and $\overline{\phi}$ respectively, we obtain the joint scattering pyramid transform.
In this implementation, 
we choose temporal wavelets $\psi(t)$ to be dyadic complex Morlet wavelets,
and $\widetilde{\psi}$ to be dyadic real Haar wavelets to preserve good frequency localization,
with no frequency downsampling. 

%The resulting second order scattering coefficients are thus
%\[
%\tilde{S} x (t, \la_1, \gamma_2,  \la_2) = 
%|F \ast \bar \Psi_{\gamma_2, \la_2} | \ast \overline{\phi}( \log_2 \la_1, t)~,
%\]
%where $\overline{\phi}(\gamma, t)$ is a two-dimensional blurring kernel with temporal 
%scale $2^J$ and log-frequency scale $2^{J_h}$. 
We can reapply the same operator as many times $k$ as desired until reaching a temporal 
context $T = \Delta_k$, but in this implementation we demonstrate the method with $k\leq 2$.
If the wavelet filters are chosen such that they define a non-expansive mapping \cite{pami}, it results that every layer defines a metric which is increasingly contracting:

$$\| |W^k| x - |W^k| x' \| \leq \| |W^{k-1}| x - |W^{k-1}| x' \| \leq \| x - x' \| ~.$$

Every layer thus produces new feature maps at a lower temporal resolution. 
In the end we obtain a tree of different representations,
$$\Phi_j(x) = |W^j| x, \quad \textrm{with} \quad j=1,\ldots, k.$$ 


%\subsection{Pyramid Wavelet Scattering}
%
%Therefor, instead of using a fixed bandwidth smoothing kernel 
%at all layers of the representation we sample each layer at critical rate. In this way, we cunstruct a pyramid of
%scattering coefficients with a different temporal resolutions at each level.
%
%
%
%These first layer coefficients give localized information both in time and frequency, 
%with a trade-off dictated by the $Q$ factor.
%
%$$|W^2| x = \{ x^1_i \ast \phi_2 (2 n)~,~| x^1_i \ast \psi_2 (2n)| \}_{i=1\dots |W^1|}~.$$
%Every layer thus produces new feature maps at a lower temporal resolution.
%
%We can reapply the same operator as many times $k$ as desired until reaching a temporal 
%context $T = 2^k \Delta_1$. If the wavelet filters are chosen such that they define a non-expansive mapping \cite{pami}, 
%it results that every layer defines a metric which is increasingly contracting:
%
%$$\| |W^k| x - |W^k| x' \| \leq \| |W^{k-1}| x - |W^{k-1}| x' \| \leq \| x - x' \| ~.$$

%
%
%\subsection{Analysis operator}
%
%Common choices are the (non-negative) spectrogram or the the constant $Q-$transform \cite{}. 
%
%Given the observed noisy signal $x$, we denote it's spectrogram as $\bb{\Phi}(x) = |\bb{X}| \in \RR^{m \times n}$ comprising $m$ frequency bins and $n$ temporal frames. $\bb{X} \in \CC^{m \times n}$ contains in it's $i-$th column the Discrete Fourier Transform (DFT) of the $i-th$ frame of $x$, $\bb{x}_i\in \RR^{m}$ ,and it is given by, $\bb{X}_i = \bb{W}^\Tr \bb{x}_i$.
%
%

\section{Source Separation Algorithm}
\label{algosec}
We show in this section how the inverse problem of source separation 
can be solved via a sparse coding in the scattering domain, followed by phase recovery.

We consider the supervised 
monoaural source separation problem, in which one observes mixtures 
\begin{equation}
\label{ssep}
x(t) = x_1(t) + x_2(t)~,
\end{equation}
where $x_i$ come from sources from which we assume training data $X_i=\{x_{ij}\}_{j \leq K}$, 
and one is asked to produce estimates $\widehat{x_i}$. 
For example, we might consider a male-vs-female separation task, by 
collecting male and female training examples. 

The supervision provides  prior information on the nature of each of the 
components. However, high-dimensional speech signals have large variability, 
most of which is uninformative for the purposes of estimating $x_i$ in (\ref{ssep}).
The training data can be exploited more efficiently in the scattering domain, since 
intra-class variability given by small pitch and timber variations is linearized up to 
larger temporal scales without loosing as much discriminative information as 
the spectrogram \cite{deepscatt,pami}.

Let $\Phi(X_i)$ be the scattering representation of the training examples of each 
source. We consider a non-linear approximation of each source using non-Negative
matrix factorization:
\begin{equation}
\min_{D_i\geq 0, Z_i\geq 0} \frac{1}{2} \| \Phi(X_i) - D_i \, Z_i \|_F^2 + \lambda \| Z_i \|_1~.
\end{equation}
This model exploits the linearization properties of scattering coefficients since it 
searches low-dimensional locally linear approximations. 

At test time, given and input $x$, $x_1$ and $x_2$ are estimated by minimizing
\begin{equation}
\label{model}
\min_{x'_i, z_i} \sum_{i=1,2} \frac{1}{2} \| \Phi(x'_i) - D_i z_i \|_2^2 + \lambda \| z_i \|_1 \quad\,s.t. ~x=x'_1 + x'_2~.
\end{equation}
Problem (\ref{model}) is minimized with an alternating gradient descent between $x'_i$ and $z_i$. 
Fixing $z_i$ and minimizing with respect to $x'_i$ requires locally inverting the scattering 
operator $\Phi$, which amounts to solve an overcomplete phase recovery problem and 
can be solved with gradient descent, as shown in \cite{bruna2013audio}. 
Fixed $x'_i$, solving for $z_i$ is a standard $\ell_1$ sparse coding problem, which can be solved 
efficiently with proximal splitting algorithms. In this work, we use the LARS algorithm. 
 
%Explain the greedy algorithm.
When the analysis operator $\Phi$ is able to produce sparse representations of the sources, 
then 
\begin{eqnarray*}
\| \Phi(x'_1) - D_1 z_1 \|_2^2 + \| \Phi(x'_2) - D_2 z_2 \|^2 &\approx& \\
\| \Phi(x'_1) + \Phi(x'_2) - \sum_{i=1,2} D_i z_i \|_2^2 &\approx& \\
 \| \Phi(x) - D_1 z_1 - D_2 z_2 \|_2^2 ~,
\end{eqnarray*}
which can be used in practice to produce a greedy initialization for (\ref{model}) as follows. 
We first obtain $\widehat{\Phi(x_i)}= D_i z^*_i$, where the $z^*_i$ are defined as 
$$z^*_i = \arg\min_{z_i} \frac{1}{2}\| \Phi(x) - \sum_{i=1,2} D_i z_i \|_2^2 +\lambda \| z_i \|_1~.$$
Since the scattering $\Phi(x)$ is defined as \\$\Phi(x)= \{ A |W_1 x | \,,\, A | W_2 | W_1 x | | \}$, 
where $A$ is the lowpass filter and $W_1$ and $W_2$ are respectively the first and 
second layer wavelet decompositions, we can produce an estimate $\widehat{x}_i$ from 
$\widehat{\Phi(x_i)}$ by using the complex phases of $W_1 x$ and $W_2 |W_1 x|$.



%
%
%\subsection{Analysis operator}
%
%Common choices are the (non-negative) spectrogram or the the constant $Q-$transform \cite{}. 
%
%Given the observed noisy signal $x$, we denote it's spectrogram as $\bb{\Phi}(x) = |\bb{X}| \in \RR^{m \times n}$ comprising $m$ frequency bins and $n$ temporal frames. $\bb{X} \in \CC^{m \times n}$ contains in it's $i-$th column the Discrete Fourier Transform (DFT) of the $i-th$ frame of $x$, $\bb{x}_i\in \RR^{m}$ ,and it is given by, $\bb{X}_i = \bb{W}^\Tr \bb{x}_i$.
%
%

\section{Experimental results}
\label{sec:experiments}


\mypara{Evaluation settings} We evaluated the proposed method in two settings: speaker-specific and multi-speaker. In the first setting 
we trained a speaker-specific model for each speaker in the mixture and tested it using sentences (from the same speakers) outside the training set. 
In the second setting, we trained a generic model on a mixed group of male and female speakers, none of which were included in the test set.
%Comparing the results obtained for these two settings we can understand the generalization and adaptation properties of the algorithm. 
All signals where mixed at 0 $dB$ and clips resampled to $16$ KHz. 

\mypara{Data sets} We used a subset of the GRID dataset \cite{cooke2006audio} for evaluating the speaker-specific setting.
For each speaker, $500$ randomly-chosen clips were used for training (around 25 minutes) and $200$ clips were used for testing.
For the multi-speaker case we used a subset of the TIMIT dataset. We adopted the standard test-train division, using all the training recordings for bulding the models
and a subset of 12 different speakers (6 males and 6 females) for testing. For each speaker we randomly chose two clips and compared
all female-male combinations (144 mixtures). 

\mypara{Evaluation measures} We used the \emph{source-to-distortion ratio} (SDR), \emph{source-to-interference ratio} (SIR), and
\emph{source-to-artifact ratio} (SAR) from the BSS-EVAL metrics \cite{vincent2006performance}. 
%
%We also computed the standard \emph{signal-to-noise ratio} (SNR).
%
%When dealing with several frames, we computed a global score (GSDR, GSIR, GSAR and GSNR) by averaging the metrics over all test clips from the same speaker and noise weighted by the clip duration.

\mypara{Training setting} We evaluated the proposed scattering NMF model with one and two layers, reffered
as \emph{scatt-NMF\textsubscript{1}} and \emph{scatt-NMF\textsubscript{2}} respectively. As a baseline we used standard NMF 
with frame lengths of 1024 samples and 50\% overlap. 
%
The dictionaries in standard NMF were chosen with $200$ and $400$ atoms for the speaker-specific and multi-speaker
settings respectively. These values were obtained using cross-validation on a few clips separated from the training as a validation set.
%
In all cases, we applied \emph{scatt-NMF} using a scattering transforms with resolution $Q_1= 32$ and $Q_2=1$.
The resulting representation had $175$ coefficients for the first level and around $2000$ for the second layer. 
%
For the single speaker case we trained dictionaries with $200$ atoms for \emph{scatt-NMF\textsubscript{1}}  and $800$ atoms for \emph{scatt-NMF\textsubscript{2}}.
While for the multi-speaker case we used $400$ atoms for \emph{scatt-NMF\textsubscript{1}}  and $1000$ atoms for \emph{scatt-NMF\textsubscript{2}}.
In all cases, the features were frame-wise normalized and we used $\lambda=0.1$. 
%
%For the SGD algorithm we used $\eta=0.1$ and minibatch of size $50$. These were obtained by trying several values of during a small number of iterations, keeping those producing the lowest error on a small validation set. 
%

\mypara{Results} Table~\ref{ta:eval} shows the results obtained for the speaker-specific and multi-speaker settings. \footnote{Audio samples are available at \url{www.cims.nyu.edu/~bruna/scatt_source_separation}.}
In all cases we observe that the one layer scattering transform outperforms the STFT in terms of SDR.
Furthermore, there is a tangible gain in including a deeper representation; \emph{scatt-NMF\textsubscript{2}} 
performs always better than \emph{scatt-NMF\textsubscript{1}}. 
As expected, the results obtained with the speaker-specific setting are better than those of the more challanging problem
of multi-speaker setting. 

We also compared the proposed approach with the speaker-specific setting discussed in \cite{Huang_DNN_Separation_ICASSP2014}. In this work
the authors investigate several alternatives of using Recurrent Neural Networks (RNN) for speech separation.
Several optimization settings are evaluated on two given speakers of the TIMIT dataset, some of which
aim at learning short-term temporal dynamics. This is a very challenging setting
due to the very small available training data (less than 10 seconds per speaker). The evaluations of \emph{scatt-NMF\textsubscript{2}} were performed using the setting provided in \cite{Huang_DNN_Separation_ICASSP2014} (with the corresponding training, developement and testing data) while their results are copied from the paper.
The proposed approach outperforms the benchmark KL-NMF, and is competitive 
with the best performing network
reported in \cite{Huang_DNN_Separation_ICASSP2014}, which was trained in a joint discriminative way, see Table~\ref{ta:eval2}.
We expect further gains by applying discriminative dictionary learning \cite{sprechmann2014supervised}.

In summary, these results confirm that inverse problems such as speech source separation 
can benefit from the properties of stable and highly discriminative non-linear representations, 
such as scattering operators. Sparse inference is able to extract more relevant information thanks 
to the stability to time-frequency deformations, while the phase recovery can still be efficiently performed
with gradient descent.

%In all the experiments with  \emph{scatt-NMF\textsubscript} , we using a greedy approximation explained in Section \ref{scattsec}. We expect that solving it explicitly would bring some additional improvement.


\begin{table}[tb]
\caption{Comparison against models with discriminative training 
and small training set. \label{ta:eval2}}
\vspace{-4.5ex}
\begin{center}
\footnotesize{
\begin{tabular}{l|c|c|c}
  \hline\hline
& SDR & SIR & SAR \\
\hline
NMF-KL     & 5.4 &   7.3 & {\bf 7.8} \\
\hline
RNN \cite{Huang_DNN_Separation_ICASSP2014} & {\bf 7.4}  &   {\bf 11.8} & 7.5  \\
\hline
\emph{scatt-NMF\textsubscript{2}} &  6.7 & 11.1  & 6.9 \\
  \hline\hline
\end{tabular}
}
\end{center}
\vspace{-4.5ex}
\end{table}


\section{Discussion}
NMF-based audio source separation techniques can be thought as applying a synthesis operator on a feature space
given by a pooled analysis operator. Leveraging recent developemnts in signal processing, we propose to substitute
%the STFT normally used in 
the first stage with a deep scattering transform. 
The obtained features are designed to capture the joint time-frequency variability of speech signals
and efficiently represent a longer temporal context. Experimental evaluation shows that using deeper representations
leads to a tangible improvement in performance in challenging source separation settings.
A natural extension of this work is to investigate the use of learned representations instead, or on top of,
the designed ones.
%
Future work includes testing more thoroughly the potential of the proposed model in combination
with convolutional neural networks, which have been very successful in other signal and image processing problems.



% a

% References should be produced using the bibtex program from suitable
% BiBTeX files (here: strings, refs, manuals). The IEEEbib.bst bibliography
% style file from IEEE produces unsorted bibliography list.
% -------------------------------------------------------------------------
\bibliographystyle{IEEEbib}

\selectfont \bibliography{refs}

\end{document}
