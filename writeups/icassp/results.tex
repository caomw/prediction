\section{Experimental results}
\label{sec:experiments}


\mypara{Evaluation settings} We evaluated the proposed method in two settings: speaker-specific and multi-speaker. In the first setting 
we trained a speaker-specific model for each speaker in the mixture and tested it using sentences (from the same speakers) outside the training set. 
In the second setting, we trained a generic model on a mixed group of male and female speakers, none of which were included in the test set.
%Comparing the results obtained for these two settings we can understand the generalization and adaptation properties of the algorithm. 
All signals where mixed at 0 $dB$ and clips resampled to $16$ KHz. 

\mypara{Data sets} We used a subset of the GRID dataset \cite{cooke2006audio} for evaluating the speaker-specific setting.
For each speaker, $500$ randomly-chosen clips were used for training (around 25 minutes) and $200$ clips were used for testing.
For the multi-speaker case we used a subset of the TIMIT dataset. We adopted the standard test-train division, using all the training recordings for bulding the models
and a subset of 12 different speakers (6 males and 6 females) for testing. For each speaker we randomly chose two clips and compared
all female-male combinations (144 mixtures). 

\mypara{Evaluation measures} As the evaluation criteria, we used the \emph{source-to-distortion ratio} (SDR), \emph{source-to-interference ratio} (SIR), and
\emph{source-to-artifact ratio} (SAR) from the BSS-EVAL metrics \cite{vincent2006performance}. 
%
%We also computed the standard \emph{signal-to-noise ratio} (SNR).
%
%When dealing with several frames, we computed a global score (GSDR, GSIR, GSAR and GSNR) by averaging the metrics over all test clips from the same speaker and noise weighted by the clip duration.

\mypara{Training setting} We evaluated the proposed scattering NMF model with one and two layers, reffered
as \emph{scatt-NMF\textsubscript{1}} and \emph{scatt-NMF\textsubscript{2}} respectively. As a baseline we used standard NMF 
with frame lengths of 1024 samples and 50\% overlap. 
%
The dictionaries in standard NMF were chosen with $200$ and $400$ atoms for the speaker-specific and multi-speaker
settings respectively. These values were obtained using cross-validation on a few clips separated from the training as a validation set.
%
In all cases, we applied \emph{scatt-NMF} using a scattering transforms with resolution $Q_1= 32$ and $Q_2=1$.
The resulting representation had $175$ coefficients for the first level and around $2000$ for the second layer. 
%
For the single speaker case we trained dictionaries with $200$ atoms for \emph{scatt-NMF\textsubscript{1}}  and $800$ atoms for \emph{scatt-NMF\textsubscript{2}}.
While for the multi-speaker case we used $400$ atoms for \emph{scatt-NMF\textsubscript{1}}  and $1000$ atoms for \emph{scatt-NMF\textsubscript{2}}.
In all cases, the features were frame-wise normalized and we used $\lambda=0.1$. 
%
%For the SGD algorithm we used $\eta=0.1$ and minibatch of size $50$. These were obtained by trying several values of during a small number of iterations, keeping those producing the lowest error on a small validation set. 
%

\mypara{Results} Table~\ref{ta:eval} shows the results obtained for the speaker-specific and multi-speaker settings respectively.
In all cases we observe that using the one layer scattering transform outperforms the STFT in terms of SDR.
Furthermore, there is a tangible gain in including a deeper representation; \emph{scatt-NMF\textsubscript{2}} 
performs always better than \emph{scatt-NMF\textsubscript{1}} 
As is to be expected, the results obtained with the speaker-specific setting are better than those of the more challanging problem
of multi-speaker setting. 

We also compared the proposed approach with the speaker-specific setting discussed in \cite{Huang_DNN_Separation_ICASSP2014}. In this work
the authors investigate several alternatives of using Recurrent Neural Networks (RNN) for speech separation.
Several optimization settings are evaluated on two given speakers of the TIMIT dataset. This is a very challenging setting
due to the very small available training data (less than 10 seconds per speaker). The evaluations of \emph{scatt-NMF\textsubscript{2}} were performed using the setting provided in \cite{Huang_DNN_Separation_ICASSP2014} (with the corresponding training, developement and testing data) while their results are copied from the paper.
The proposed approach significantly outperforms the benchmark Kullback-Leiber NMF, and is competitive 
with the best performing network
reported in \cite{Huang_DNN_Separation_ICASSP2014}, which was trained in a joint discriminative way, see Table~\ref{ta:eval2}.
We expect further gains by applying discriminative dictionary learning \cite{sprechmann2014supervised}.

In summary, these results confirm that inverse problems such as speech source separation 
can benefit from the properties of stable and highly discriminative non-linear representations, 
such as scattering operators. Sparse inference is able to extract more relevant information thanks 
to the stability to time-frequency deformations, while the phase recovery can still be efficiently performed
with gradient descent.

%In all the experiments with  \emph{scatt-NMF\textsubscript} , we using a greedy approximation explained in Section \ref{scattsec}. We expect that solving it explicitly would bring some additional improvement.


\begin{table}[tb]
\caption{Comparison against models with discriminative training 
and small training set. \label{ta:eval2}}
\vspace{-4.5ex}
\begin{center}
\footnotesize{
\begin{tabular}{l|c|c|c}
  \hline\hline
& SDR & SIR & SAR \\
\hline
NMF-KL     & 5.4 &   7.3 & {\bf 7.8} \\
\hline
RNN \cite{Huang_DNN_Separation_ICASSP2014} & {\bf 7.4}  &   {\bf 11.8} & 7.5  \\
\hline
\emph{scatt-NMF\textsubscript{2}} &  6.7 & 11.1  & 6.9 \\
  \hline\hline
\end{tabular}
}
\end{center}
\vspace{-4.0ex}
\end{table}


\section{Discussion}
NMF-based audio source separation techniques can be thought as applying a synthesis operator on a feature space
given by a pooled analysis operator. Leveraging recent developemnts in signal processing, we propose to substitute
%the STFT normally used in 
the first stage with a deep scattering transform. 
The obtained features are designed to capture the joint time-frequency variability of speech signals
and efficiently represent a longer temporal context. Experimental evaluation showed that using deeper representations
leads to a tangible improvement in performance in challenging source separation settings.
A natural extension of this work would be to investigate the use of learned representations instead (or on top of)
the designed ones.
%
Future work includes testing more thoroughly the potential of the proposed model in combination
with convolutional neural networks, which have been very successful in other signal and image processing problems.

