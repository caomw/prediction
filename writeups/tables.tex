% !TEX TS-program = pdflatex
% !TEX encoding = UTF-8 Unicode


\documentclass[11pt]{article} % use larger type; default would be 10pt

\usepackage[utf8]{inputenc} % set input encoding (not needed with XeLaTeX)


\usepackage{geometry} % to change the page dimensions
\geometry{letterpaper} % or letterpaper (US) or a5paper or....
% \geometry{margin=2in} % for example, change the margins to 2 inches all round
% \geometry{landscape} % set up the page for landscape
%   read geometry.pdf for detailed page layout information

\usepackage{graphicx} % support the \includegraphics command and options

% \usepackage[parfill]{parskip} % Activate to begin paragraphs with an empty line rather than an indent

%%% PACKAGES
\usepackage{booktabs} % for much better looking tables
\usepackage{array} % for better arrays (eg matrices) in maths
\usepackage{paralist} % very flexible & customisable lists (eg. enumerate/itemize, etc.)
\usepackage{verbatim} % adds environment for commenting out blocks of text & for better verbatim
\usepackage{subfig} % make it possible to include more than one captioned figure/table in a single float
% These packages are all incorporated in the memoir class to one degree or another...

%%% HEADERS & FOOTERS
\usepackage{fancyhdr} % This should be set AFTER setting up the page geometry
\pagestyle{fancy} % options: empty , plain , fancy
\renewcommand{\headrulewidth}{0pt} % customise the layout...
\lhead{}\chead{}\rhead{}
\lfoot{}\cfoot{\thepage}\rfoot{}

%%% SECTION TITLE APPEARANCE
\usepackage{sectsty}
\allsectionsfont{\sffamily\mdseries\upshape} % (See the fntguide.pdf for font help)
% (This matches ConTeXt defaults)

%%% ToC (table of contents) APPEARANCE
\usepackage[nottoc,notlof,notlot]{tocbibind} % Put the bibliography in the ToC
\usepackage[titles,subfigure]{tocloft} % Alter the style of the Table of Contents
\renewcommand{\cftsecfont}{\rmfamily\mdseries\upshape}
\renewcommand{\cftsecpagefont}{\rmfamily\mdseries\upshape} % No bold!

% additional packages for math typesetting
\usepackage{amsmath}
\usepackage{amsfonts}
\usepackage{fixltx2e}

\newcommand{\R}{\mathbb{R}}
\newcommand{\W}{\mathcal{W}}
\newcommand{\Rel}{\mathbb{Z}}
\newcommand{\Int}{\mathbb{N}}
\newcommand{\Eff}{\mathcal{E}}
\newcommand{\Compl}{\mathscr{C}}

\title{Prediction and Inverse Problems in Dynamical Systems}
\author{Joan Bruna, Pablo Sprechmann}
%\date{} % Activate to display a given date or no date (if empty),
         % otherwise the current date is printed 

\begin{document}
\maketitle


\section{Results}

\begin{table*}[t]
\centering
\begin{tabular}{l|c|c|c || c |c |c }
  \hline\hline
  & \multicolumn{3}{c||}{Speaker-Specific} & \multicolumn{3}{c}{Multi-Speaker} \\
  \hline
 & SDR & SIR & SAR & SDR & SIR & SAR\\
\hline
NMF  &8.8 [2] & 18.6 [3] &   9.6 [1.6] & 6.1 [2.9] &   14.1 [3.8] & 7.4 [2.1] \\
\hline
\emph{scatt-NMF\textsubscript{1}} & 10.3 [2.0]  & 19.7 [3.3]  & 11.0 [1.7] &  6.2 [2.8] &   13.5 [3.5] & 7.8 [2.2] \\
\emph{scatt-NMF\textsubscript{2}} &  {\bf 10.6} [1.8] & {\bf 20.5} [3.0]  & {\bf 11.3} [1.7]  &  {\bf 6.9} [2.7] & {\bf 16.0} [3.5]  & {\bf 7.9} [2.2] \\
  \hline
  \hline
\end{tabular}
\caption{Separation with speakers-specific and multi-speaker settings. Average SDR, SIR and SAR (in $dB$) for NMF and proposed  and \emph{scatt-NMF\textsubscript{2}}. Standard deviation of each result shown between brackets. \label{ta:eval}}
\vspace{-2ex}
\end{table*}


\begin{table*}[t]
\centering
\begin{tabular}{|l|c|c|c |}
  \hline\hline
     \multicolumn{4}{|c|}{SDR}\\
  \hline
\,  & F1 & F2 & F3 \\
\hline
M1 & 10.4  &  8.6  &  8.8\\
M2 &  10.3 &   8.8 &   7.7\\
M3 &  10.2 &  7.9 &   7.9 \\
  \hline
  \hline
\end{tabular}
\hspace{1ex}
\begin{tabular}{|l|c|c|c |}
  \hline\hline
    \multicolumn{4}{|c|}{SAR}\\
\hline
\,  & F1 & F2 & F3 \\
\hline
M1 & 11.2 &   9.7  &  9.9\\
M2 &  11.0  &  9.5 &  9.0\\
M3 &  10.9 & 9.1 &  9.1\\
  \hline
  \hline
\end{tabular}
\hspace{1ex}
\begin{tabular}{|l|c|c|c |}
  \hline\hline
  \multicolumn{4}{|c|}{SIR}\\
  \hline
\,  & F1 & F2 & F3 \\
\hline
M1 & 19.5 & 16.1 & 16.4\\
M2 &  19.5 & 15.6 &  14.3\\
M3 &  19.6 &  15.2 & 14.9\\

  \hline
  \hline
\end{tabular}\\
\caption{SDR \label{ta:eval}}
\end{table*}




\end{document}






