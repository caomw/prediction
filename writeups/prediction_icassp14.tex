% !TEX TS-program = pdflatex
% !TEX encoding = UTF-8 Unicode


\documentclass[11pt]{article} % use larger type; default would be 10pt

\usepackage[utf8]{inputenc} % set input encoding (not needed with XeLaTeX)


\usepackage{geometry} % to change the page dimensions
\geometry{letterpaper} % or letterpaper (US) or a5paper or....
% \geometry{margin=2in} % for example, change the margins to 2 inches all round
% \geometry{landscape} % set up the page for landscape
%   read geometry.pdf for detailed page layout information

\usepackage{graphicx} % support the \includegraphics command and options

% \usepackage[parfill]{parskip} % Activate to begin paragraphs with an empty line rather than an indent

%%% PACKAGES
\usepackage{booktabs} % for much better looking tables
\usepackage{array} % for better arrays (eg matrices) in maths
\usepackage{paralist} % very flexible & customisable lists (eg. enumerate/itemize, etc.)
\usepackage{verbatim} % adds environment for commenting out blocks of text & for better verbatim
\usepackage{subfig} % make it possible to include more than one captioned figure/table in a single float
% These packages are all incorporated in the memoir class to one degree or another...

%%% HEADERS & FOOTERS
\usepackage{fancyhdr} % This should be set AFTER setting up the page geometry
\pagestyle{fancy} % options: empty , plain , fancy
\renewcommand{\headrulewidth}{0pt} % customise the layout...
\lhead{}\chead{}\rhead{}
\lfoot{}\cfoot{\thepage}\rfoot{}

%%% SECTION TITLE APPEARANCE
\usepackage{sectsty}
\allsectionsfont{\sffamily\mdseries\upshape} % (See the fntguide.pdf for font help)
% (This matches ConTeXt defaults)

%%% ToC (table of contents) APPEARANCE
\usepackage[nottoc,notlof,notlot]{tocbibind} % Put the bibliography in the ToC
\usepackage[titles,subfigure]{tocloft} % Alter the style of the Table of Contents
\renewcommand{\cftsecfont}{\rmfamily\mdseries\upshape}
\renewcommand{\cftsecpagefont}{\rmfamily\mdseries\upshape} % No bold!

% additional packages for math typesetting
\usepackage{amsmath}
\usepackage{amsfonts}


\newcommand{\R}{\mathbb{R}}
\newcommand{\W}{\mathcal{W}}
\newcommand{\Rel}{\mathbb{Z}}
\newcommand{\Int}{\mathbb{N}}
\newcommand{\Eff}{\mathcal{E}}
\newcommand{\Compl}{\mathscr{C}}

\title{Prediction and Inverse Problems in Dynamical Systems}
\author{Joan Bruna, Pablo Sprechmann}
%\date{} % Activate to display a given date or no date (if empty),
         % otherwise the current date is printed 

\begin{document}
\maketitle


\section{Introduction}


$x$ is the input speech signal. 
We define $\Phi(x)$ a pooling analysis operator, which can be
either the spectrogram, the scattering, or something that we learn. 
It can have several layers. 

Then, given training examples $X=(x_i)_i$, we learn a synthesis dictionary $D$
on $\Phi(X)$. 


What is the role of the analysis operator $\Phi$? 
Its role is to perform a contraction of each speaker's class, so that each source 
is embedded into a smooth manifold, which is then encoded using sparse coding. 
 It needs to preserve discriminability between sources. 
We can also specialize the analysis operators for each class (TODO: why? what do we gain 
by that?). 

We will restrict ourselves to analysis operators of the form 
$\Phi(x) = | W x |$, where $W$ is overcomplete, and close to a tight frame.

Variations of the model. There are several options for choosing the analysis $\Phi$ 
and several options for choosing the synthesis. 



Estimation algorithm. 
Given $y=x_1 +  x_2$, we construct estimates $\hat{x_1}$ and $\hat{x_2}$ 
by solving
\begin{equation}
\label{estalgo}
\arg\min_{x'_i, z_i} \| y - x'_1 - x'_2 \|^2 + \sum_{i=1,2} \| \Phi(x'_i) - D_i z_i \|^2 + \lambda \| z_i \|~.
\end{equation}
A priori, this algorithm does not require computing the modulus/phase of the observation $y$. 
An initialization of this algorithm, in the case where $\Phi(x)= | W x |$, assumes that 
$$\Phi(y)_k^2 \approx \Phi(x_1)_k^2 + \Phi(x_2)_k^2~,$$
and then estimates $\Phi(x_i)$ independently using each dictionary. Finally, $\hat{x}_i$ are obtained
by using the phase of $Wy$.

In terms of the model, we need to decide which analysis operator to use, and 
how to train the dictionaries. 

Alternative: Use a synthesis+pooling operator combined with analysis. In other words, 

Definition of $\Phi$. We have verified empirically that constant $Q$-transform yields better 
separation than STFT as the first layer. Now, we have several options:
\begin{itemize}
\item $\Phi_1(x) = Sc_1 x $ (CQT (1 layer scattering). 
\item $\Phi_2(x) = F_1 x$ (STFT)
\item $\Phi_3(x) = Sc_2 Sc_1 x$ (2 layer scattering)
\item $\Phi_4(x)= |A_2| Sc_1 x$ (1 layer scattering followed by learnt scattering). 
\item $\Phi_5(x)= |S_2| Sc_1 x$ (1 layer scattering followed by group lasso). 
%\item $\Phi_5(x) =  Sc_1 x$ 1 layer scattering followed by  group lasso.
\end{itemize}

$\Phi_4$ is more stable than $\Phi_5$, but it may produce coefficients harder to separate 
in the final layer. 

One layer vs two layers: A major difference is the temporal context. Two layer operators can
extract temporal coherence and exploit temporal dynamics on a larger range than first layer operators. 

Definition of the discriminative layer: We will stick to standard Dictionary learning.

There are several levels of supervision:
\begin{itemize}
\item Male vs Female: one dictionary for male and another for female.
\item Unsupervised: A single ``speech" dictionary. 
\item Speaker vs ``rest of the world". We prioritize one speaker and consider the rest to be noise. 
\end{itemize}



Planning:
\begin{enumerate}
\item Move to the setting with various speakers. The exhaustive test
is 3 levels of supervision and 2 datasets (Timit and Grid). [P, 9/30]
\item In priority, we try: TIMIT denoising and TIMIT speaker separation (Male vs Female). [P, 9/30]
\item Test $\Phi_1$ vs $\Phi_2$ vs $\Phi_3$. [J, 10/1-2]
\item Implement the estimation algorithm (\ref{estalgo}) [J, 10/1-2].
\item Optimize temporal context. This is governed by $\Phi$. [P, 10/1-2].
\item Implement the learning of analysis operator $A(x) = |Wx|$. Perhaps start with 
the algorithm from paper ??. What criteria??
\item If we have time, compare also $\Phi_4$.
\item Maths: explain why analysis is a better idea than synthesis for $\Phi$. 
\item Writeup. 
\item Discriminative fine-tuning (probably out of time). 
\end{enumerate}


\section{Previous work}


\section{Spatio-temporal Pooling and Inhibition}


\section{Unsupervised Learning}

We describe first the case with two known speakers. 
We train two dictionaries $\mathcal{D}_i$, $i=1,2$ using 
the following model:

\begin{equation}
\| x - \mathcal{D} z \|^2 + \phi_{\mathcal{G}}(z) 
\end{equation}
with $\phi_{\mathcal{G}}$ being a spatio-temporal group lasso. 
 Denoting ${z}^p = \phi_{\mathcal{G}}(z) $, 
the second layer is
\begin{equation}
\|{z}^p  - \widetilde{\mathcal{D}} \tilde{z} \|^2 + \phi_{\mathcal{G}}(\tilde{z}) 
\end{equation}


\section{Discriminative Training with Bi-level}

\subsection{Inference}
Given $x=x_1 + x_2$, we want to infer $x_1$ and $x_2$. 
We use the following scheme:

\begin{equation}
(z_1^*, z_2^*) = arg \min \| x - [\mathcal{D}_1 \mathcal{D}_2] [z_1 ; z_2] \|^2 + \phi_{\mathcal{G}}(z_1)  + \phi_{\mathcal{G}}(z_2) 
\end{equation}
\begin{equation}
(\tilde{z}_1^*, \tilde{z}_2^*) = arg \min \|  - [\mathcal{D}_1 \mathcal{D}_2] [z_1 ; z_2] \|^2 + \phi_{\mathcal{G}}(z_1)  + \phi_{\mathcal{G}}(z_2) 
\end{equation}




\section{Experimental Results}







\end{document}






